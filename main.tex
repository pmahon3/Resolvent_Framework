\documentclass{article}
\usepackage[left=2cm, right=2cm]{geometry}
\usepackage[parfill]{parskip}
\usepackage{graphicx}
\usepackage{amsmath} 
\usepackage{amsthm}
\newtheorem{definition}{Definition}
\usepackage{amssymb}
\usepackage{mathtools}
\usepackage[colorlinks]{hyperref}
\usepackage[symbols,nogroupskip,sort=none]{glossaries-extra}
\renewcommand*{\glspostdescription}{\medskip}

\newcommand{\N}{\mathbb{N}}
\newcommand{\Z}{\mathbb{Z}}
\newcommand{\Q}{\mathbb{Q}}
\newcommand{\R}{\mathbb{R}}
\newcommand{\C}{\mathbb{C}}
\newcommand{\G}{\mathbb{G}}
\newcommand{\X}{\text{x}}
\newcommand{\W}{\text{w}}
\newcommand{\Rho}{\mathrm{P}}

\glsxtrnewsymbol[description={Banach space $(X, ||\cdot||$)}]{X}{$X$}

\glsxtrnewsymbol[description={Weighting kernel at $x_t$, parameterized by $\theta \in \mathbb{R}, x_t \in X$}]{kernel}{$w(x, x_t, \theta): X \times X \times \mathbb{R} \rightarrow \mathbb{R}$}

\begin{document}
\title{A Resolvent Framework for Global and Local Nonlinear Semigroups}
\author{Patrick S. Mahon}
\maketitle

\tableofcontents
\printunsrtglossary[type=symbols,style=long]

\section{Definitions}
\begin{definition}
The process-$t$ matrix, read  ``process until $t$ matrix'' or more simply ``process until $t$'', is 
$$ X_t = 
\begin{bmatrix}
    x_t \\ 
    x_{t-h} \\ 
    x_{t-2h} \\
    \vdots
\end{bmatrix}
$$
where rows belong to $X$.
\end{definition}

\begin{definition}
The $t$-weighting matrix of the process-$i$ is given by $$W(X_i, x_t, \theta) =
\begin{bmatrix}
    w(x_j, x_t, \theta) & 0 & 0 &\cdots \\
    0 & w(x_{j-h}, x_t, \theta) & 0 &  \cdots\\
    0 & 0 & w(x_{j-2h}, x_t, \theta) & \cdots\\
    \vdots & \vdots & \vdots & \ddots
\end{bmatrix}$$\medskip
$W_t^\theta X_{i}$ is short hand for the the product $W(X_i, x_t, \theta) X_i$ 
\begin{align*}
    W_t^\theta X_{i} :&= W(X_i, x_t, \theta) \cdot X_i = 
    \begin{bmatrix}
        w(x_i, x_t, \theta) \cdot x_i \\
        w(x_{i-h}, x_t, \theta) \cdot x_{i-h} \\
        w(x_{i-2h}, x_t, \theta) \cdot x_{i-2h} \\
        \vdots
\end{bmatrix}
\end{align*}
which is the $t$-weighting of the process-$i$.
\end{definition}

\section{Introduction}

\subsection{$C_t$ as a globally linear map}
Consider the map 
$$
    C: x_t \mapsto x_{t+h}
$$
for $x_i \in X_t \subset X$ and
\begin{align*}
    x_{t+h} & = C x_t
\end{align*}
C may be state dependent, $C(x(t))$, non-autonomous, $C(t)$, or both, C(t, x(t)). We denote the possibility of any such case as $C_t$.

If $C_t$ is globally linear then
\begin{align*}
    X_{t+h} = X_t C_t
\end{align*}
which is solved via,
\begin{align*}
    C_t = X_{t}^{-1} X_{t+h}
\end{align*}
In an applied time series setting this solution for $C_t$ is the \textit{auto-regressive} or $AR$ model for the process $X_t$.

\subsection{$C_t^\theta$ as a locally linear map}
If $C$ is not globally linear than the $t$-weighting can be introduced such that
\begin{align*}
    W_t^\theta X_{t+h} & = W_t^\theta X_t C_t^\theta \\
\end{align*}
where the weighting kernel, $w$, of  $W_t^\theta$ is parameterized by $\theta$. The exact local weighting at $x_t$ is achieved as,
\begin{align*}
    \lim_{\theta\rightarrow\infty} w(x_i, x_t, \theta) = \delta(x_i - x_t)
\end{align*}
for all $x_i \in X_t$. Typically the kernel is chosen so $w \sim e^{-\theta}$. The solution for $C_t^\theta$ is
\begin{align*}
    C_t^\theta &= {(W_t^\theta X_t)}^{-1} W_t^\theta X_{t+h} \\
    &= X_t^{-1} (W_t^\theta)^{-1} W_t^\theta X_{t+h}
\end{align*}
Taking the limit, $W_t^\theta$ reduces to,
\begin{align*}
    \lim_{\theta\rightarrow\infty} W_t^\theta = 
    \begin{cases} 
      \delta(x_i - x_t) = 1 &\text{ if }i=j \wedge x_i = x_t\\
      0 &\text{ if } i\neq j \vee x_i \neq x_t
   \end{cases}
\end{align*}
which is a diagonal matrix whose only non-zero entries are $1$'s corresponding to states arbitrarily close to $x_t$. If $X_t$ never returns to states arbitrarily close to $x_t$ then $W_t^\theta$ reduces to the rank-1 matrix with a single non-zero entry, 
$$
    (W_t^\theta)_{1,1} = 1
$$

If $X_t$ is periodic at frequency $k$ the set of all indices $i$ on the diagonal where $(W_t^\theta)_{i=j} = 1$, is 
$$ 
    \{W_t^\theta\}_1 = \{n\in\N: i = 1+nk \}
$$

If $X_t$ is ergodic than the rank depends on the nature and frequency of close returns to neighbourhoods containing $x_t$, e.g. 
$$
    U = \{x\in X, \delta = a \in \R: w(x, x_t, \theta) < \delta \}
$$ 
could be used to define ``close''.

\section{One Parameter Semigroups}
Placing $C_t^\theta$ in the theory of semigroups can help us understand the relationship between locality, $\theta$, and the potential higher order terms for some generating process of the elements of $X_t$. The study of one parameter semigroups provides useful tools that connect the operators which generate the semigroup to the semigroups themselves, by way of \textit{generating theorems}. We start by generalizing $C_t$ as a semigroup.

\subsection{$C_t$ as a one parameter semigroup}

We define $C_t$
\begin{align*}
    C(t)&: t \rightarrow \G \nonumber &t\in\R_{\geq0}
\end{align*}
where we take the group $\G$ as $\{X_t\}_{t\geq0}$ 
\begin{align*}
    C(t)&: t \rightarrow X_{t+h} \nonumber &X_{t+h}\in X
\end{align*}
with $(X, ||\cdot||)$ being Banach. 

$\{C(t)\}_{t\geq 0}$ forms a one parameter semigroup if the the following requirements are met:

\qquad Associativity:
\begin{equation}
    \forall t,s\geq0: C(t+s) = C(t)C(s) \label{eq:associativity}
\end{equation}
\qquad Identity: 
\begin{equation}
    C(0) = I \label{eq:identity}
\end{equation}

\subsection{$C_0$-semigroup}

Continuity for one parameter semigroups are often defined by the following properties:

\qquad Strongly Continuous: \\

\qquad\qquad For the \textit{infinitesimal generator}, $A$, defined on the domain $D(A)$
\begin{equation}
    Ax =  \lim_{t\downarrow0} \frac{1}{t} (C(t) - I)x \label{eq:continuity}
\end{equation}
\qquad\qquad the limit must exist. 

The strongly continuous one parameter semigroups is denoted as the $C_0$-semigroup. This semigroup provides a generalization of the exponential in the case where it is also uniformly continuous;

\qquad Uniform continuity:

\qquad\qquad
\begin{align}
    &{\lim_{t \downarrow 0} || C(t) x(0) - x(0) || \rightarrow 0}  &\forall x(0) \in X \label{eq:uniformity}
\end{align}

The linear operator $A$ is the infinitesimal generator of a uniformly continuous semigroup if and only if $A$ is a bounded linear operator. In this case the infinitesimal generator $A$ of the semigroup $C(t)$ must satisfy
$$
    C(t) = e^{At} := \sum_0^{\infty} \frac{A^k}{k!} t^k \label{eq:generator}
$$
However if $C(t)$ is strongly but not uniformly continuous then $A$ is not bounded and $e^{At}$ need not converge.

The $C_0$-semigroup provides a good, however insufficient, basis for understanding the relationships of locality and the higher order terms of the generator for $X_t$. 
\subsection{$C_0^n$-semigroup}

 $C_0$-semigroup to be a special case of a more general strongly continuous one parameter semigroup, the nonlinear $C^n_0$-semigroup, whose infinitesimal generator expresses higher order effects in the limit of ever smaller finite differences. 
 
 We both guide and justify the definition such a semigroup on the basis that the exponential of the infinitesimal generator yields a well known generalization of the Taylor series that uses the infinitesimal limit of finite differences. This is the Hille series. 
 
The upshot is that the $C^n_0$-semigroup and it's infinitesimal generator can carry intuition regarding continuity and nonlinearity over to pragmatic operator theory approaches, wherein we are better suited to investigate locality.

\subsubsection{The Hille Operator for nonlinear semigroups}

We can relate the forward differences to the map of $C$ where,
$$
    \Delta_h^1 x(t) = x(t+h) - x(t) = C_h^1 x(t) - x(t)\\
$$. 

For higher order differences the difference operator series is
\begin{align}
    \Delta_h^n x(t) &= \sum_{k=0}^n (-1)^{n-k} {n \choose k} x(t+kh) \nonumber \\
    &= \sum_{k=0}^n (-1)^{n-k} {n \choose k} C_h^k x(t) \label{eq:delta_c}
\end{align}

If we define the $n$-order generator,
$$
    A_h^n = \frac{\Delta_h^n}{h^n}
$$

The the $n$-order generators of the semigroup, $A_h^n$, describe higher-order terms of the expansion when viewed as finite-difference corrections to the first-order dynamics, $A_h^1$, capturing more detailed behavior of the system.

A semigroup can be defined over the sum of the $n$-order generators,
\begin{align}
    C_h(t) = e^{A_h t} &:= \sum_{n=0}^\infty \frac{t^n}{n!} A^n_h \nonumber\\
    &:= \sum_{n=0}^\infty \frac{t^n}{n!} \frac{\Delta_h^n}{h^n} \label{eq:hille_operator}
\end{align}

In the limit of difference size $h$ we get
\begin{align*}
    \lim_{h\rightarrow0^+} C_h(t) &= \lim_{h\rightarrow0^+} \sum_{n=0}^\infty \frac{t^n}{n!} \frac{\Delta_h^n}{h^n}\\
    H(t) &= \lim_{h\rightarrow0^+} \sum_{n=0}^\infty \frac{t^n}{n!} \frac{\Delta_h^n}{h^n}
\end{align*}
where we call $H(t)$ the \textit{Hille operator}. On application to $x(t)$, this operator yields the Hille series expansion
\begin{align*}
    H(t) x(t) &= \lim_{h\rightarrow0^+} \sum_{n=0}^\infty \frac{t^n}{n!} \frac{\Delta_h^n x(t)}{h^n} 
\end{align*}

Thus, the Hille Operator provides a way to generalize the semigroup dynamics to include higher-order effects via the Hille Series, defining a strongly and uniformly continuous one parameter nonlinear semigroup. We denote the order of the semigroup as $C_0^n$-semigroup. Where $n$ is non finite we have the strongly nonlinear $C_0^\infty$-semigroup. 

We can reintroduce (\ref{eq:delta_c}) so,
\begin{align*}
    H(t)  &= \lim_{h\rightarrow0^+} \sum_{n=0}^\infty \frac{t^n}{n!} \frac{\Delta_h^n}{h^n} \\
    &= \lim_{h\rightarrow0^+} \sum_{n=0}^\infty \frac{t^n}{n!} \frac{ \sum_{k=0}^n (-1)^{n-k} {n \choose k} C_h^k}{h^n}
\end{align*}

\subsection{The resolvent of $C^n_0$}

The resolvent 
$$
    R(\lambda, A) = (\lambda I - A)^{-1}
$$
is an important because it tells us whether the operator $A$ has an inverse for certain values of $\lambda$. This can help define the spectrum and growth behavior of the semigroup.

For
\begin{align*}
    H(t) &= \sum_{n=0}^\infty \frac{t^n}{n!} A^n_h \nonumber\\
    &= \lim_{h\rightarrow0^+} \sum_{n=0}^\infty \frac{t^n}{n!} \frac{\Delta_h^n}{h^n}
\end{align*}
we are interested in the resolvent set that provides a well definied inverse,
$$
    \rho(A^n) = \left\{ n\in\Z, \lambda\in\C: \lim_{h\rightarrow0^+} \left( \lambda I - A^n_h \right) ^{-1}\right\}
$$ 
We have
\begin{align*}
     R(\lambda, A^n) &= \lim_{h\rightarrow0^+}\left( \lambda I - A^n_h\right) ^{-1}\\
      &= \lim_{h\rightarrow0^+}\left( \lambda I - \frac{\Delta^n_h}{h^n} \right) ^{-1}
\end{align*}
so,
\begin{align*}
    \lim_{h\rightarrow0^+} \left( \lambda I- \frac{\Delta^n_h}{h^n}\right)x(t) = y(t)
\end{align*}
where,
\begin{align*}
     0 &\neq \lim_{h\rightarrow0^+} \left| \lambda I- \frac{\Delta^n_h}{h^n} \right| \\
      0 &\neq \lim_{h\rightarrow0^+} \left| \lambda I - \frac{ \sum_{k=0}^n (-1)^{n-k} {n \choose k} C_h^k}{h^n} \right|
\end{align*}
So in short the members of the resolvent set cannot belong to the spectrum of the $n$ order finite difference approximations of the nonlinear generator,
\begin{align*}
    A^h = \sum_{n} \frac{\sum_{k=0}^n (-1)^{n-k} {n \choose k} C_h^k}{h^n}
\end{align*}
Therefor
$$
    \rho(A_h) = \bigcap_n \rho(A^n_h) = \C / \left\{\bigcup_n \sigma(A^n_h)\right\}
$$ 
where $\sigma(A^n_h)$ is the spectrum of the $n$ order finite difference approximation.
\end{document}
\documentclass{article}
\usepackage{geometry}
\usepackage[parfill]{parskip}
\usepackage{graphicx}
\usepackage{amsmath} 
\usepackage{amsthm}
\newtheorem{definition}{Definition}
\usepackage{amssymb}
\usepackage{mathtools}
\usepackage{tcolorbox}
\usepackage[colorlinks]{hyperref}
\usepackage[symbols,nogroupskip,sort=none]{glossaries-extra}
\renewcommand*{\glspostdescription}{\medskip}

\newcommand{\N}{\mathbb{N}}
\newcommand{\Z}{\mathbb{Z}}
\newcommand{\Q}{\mathbb{Q}}
\newcommand{\R}{\mathbb{R}}
\newcommand{\C}{\mathbb{C}}
\newcommand{\G}{\mathbb{G}}
\newcommand{\X}{\text{x}}
\newcommand{\W}{\text{w}}
\newcommand{\Rho}{\mathrm{P}}
\newcommand{\Lgr}{\mathcal{L}}
\mathchardef\Re="023C

\glsxtrnewsymbol[description={Banach space $(X, ||\cdot||$)}]{X}{$X$}

\glsxtrnewsymbol[description={Weighting kernel at $x_t$, parameterized by $\theta \in \mathbb{R}, x_t \in X$}]{kernel}{$w(x, x_t, \theta): X \times X \times \mathbb{R} \rightarrow \mathbb{R}$}

\begin{document}
\title{A Resolvent Framework for Global and Local Nonlinear Semigroups}
\author{Patrick S. Mahon}
\maketitle

\tableofcontents
\printunsrtglossary[type=symbols,style=long]

\section{Definitions}
\begin{definition}
$X_t$ is a matrix of a sequence of elements, $x \in X$, indexed by $t\geq0$
$$ X_t = 
\begin{bmatrix}
    x_t \\ 
    x_{t-h} \\ 
    x_{t-2h} \\
    \vdots
\end{bmatrix}
$$
\end{definition}

\begin{definition}
The $t$-weighting matrix of $X_i$ is given by $$W(X_i, x_t, \theta) =
\begin{bmatrix}
    w(x_j, x_t, \theta) & 0 & 0 &\cdots \\
    0 & w(x_{j-h}, x_t, \theta) & 0 &  \cdots\\
    0 & 0 & w(x_{j-2h}, x_t, \theta) & \cdots\\
    \vdots & \vdots & \vdots & \ddots
\end{bmatrix}$$\medskip
$W_t^\theta X_{i}$ is short hand for the the product $W(X_i, x_t, \theta) X_i$ 
\begin{align*}
    W_t^\theta X_{i} :&= W(X_i, x_t, \theta) \cdot X_i = 
    \begin{bmatrix}
        w(x_i, x_t, \theta) \cdot x_i \\
        w(x_{i-h}, x_t, \theta) \cdot x_{i-h} \\
        w(x_{i-2h}, x_t, \theta) \cdot x_{i-2h} \\
        \vdots
\end{bmatrix}
\end{align*}
\end{definition}

\section{Introduction and Motivation}

\subsection{$C$ as a globally linear map}
Consider the map 
\begin{align*}
    C(\cdot)&: X \rightarrow X\\
\end{align*}
$C$ may describe a time evolving map
\begin{align*}
    C(\cdot)&: x_t \mapsto x_{t+h}    &h\geq0,\,x_i \in X
\end{align*}
and may be state dependent, $C(x)$, non-autonomous, $C(t)$, or both, $C(t, x)$. If, for the autonomous state-independent case
\begin{align*}
    x_{t+h} & = C x_t
\end{align*}
$C$ is globally linear, then,
\begin{align*}
    X_{t+h} = X_t C\\
    C = X_{t}^{-1} X_{t+h}
\end{align*}
provides the solution. 

In an applied time series setting, this corresponds to an auto-regressive (AR) model for the process \( X_t \), where future states depend linearly on past states through the operator \( C \). The AR model is a discrete-time example of how a system evolves over time.


\subsection{$C_t^\theta$ as a locally linear map}
If \( C \) is not globally linear, it must exhibit either non-autonomous behavior or state dependence. In many empirical settings, implicit non-autonomy arises naturally through state dependence, where \( C(x(t)) \) depends on the system's evolving state, effectively introducing time dependence via the state trajectory.

To try to account for this dependence we can introduce the $t$-weighting to the AR solution,
\begin{align*}
    W_t^\theta X_{t+h} & = W_t^\theta X_t C_t^\theta \\
    C_t^\theta & = (W_t^\theta X_t)^{-1} W_t^\theta X_{t+h} \\
\end{align*}
To capture local linearity, the weighting kernel \( w(x_i, x_t, \theta) \), parameterized by \( \theta \), emphasizes states close to \( x_t \). As \( \theta \to \infty \), the local weighting becomes exact, with:
$$
    \lim_{\theta\rightarrow\infty} w(x_i, x_t, \theta) = \delta(x_i - x_t),
$$
where the Dirac delta function selects only states arbitrarily close to \( x_t \). This applies to each \( x_i \in X_t \), localizing the influence of nearby states.

Typically the kernel is chosen so $w \sim e^{-\theta}$. 
Taking the limit, $W_t^\theta$ reduces to,
\begin{align*}
    \lim_{\theta\rightarrow\infty} W_t^\theta = 
    \begin{cases} 
      \delta(x_i - x_t) = 1 &\text{ if }i=j \wedge x_i = x_t\\
      0 &\text{ if } i\neq j \vee x_i \neq x_t
   \end{cases}
\end{align*}
which is a diagonal matrix whose only non-zero entries are $1$'s corresponding to states arbitrarily close to $x_t$. If $X_t$ never returns to states arbitrarily close to $x_t$ then $W_t^\theta$ reduces to the rank-1 matrix with a single non-zero entry, 
$$
    (W_t^\theta)_{1,1} = 1
$$

If \( X_t \) exhibits periodicity with frequency \( k \), then only every \( k \)-th diagonal element of \( W_t^\theta \) remains non-zero, reflecting the periodic return to the same state:
$$
    (W_t^\theta)_{i,j} = 
    \begin{cases}
        1  &\text{ if } i=j=t+nk,\, n\in\N\\
        0  &\text{ otherwise}
    \end{cases}
$$
For ergodic processes, the rank of \( W_t^\theta \) reflects the frequency and nature of returns to neighborhoods around \( x_t \), with proximity defined by:
$$
    U = \{x \in X, \delta \in \mathbb{R}: w(x, x_t, \theta) < \delta\}
$$
This provides a natural way to define "closeness" in ergodic settings.

\section{One Parameter Linear and Nonlinear Semigroups}
Semigroup theory provides a powerful framework for understanding the evolution of systems, especially when dealing with local linearity and the influence of higher-order terms. By introducing one-parameter semigroups, we gain tools to connect the operators that generate the semigroup with the semigroup's overall dynamics. This helps us explore the relationship between locality (controlled by \( \theta \)) and the time evolution of the elements in \( X_t \).


\subsection{$C(h)$ as a one parameter semigroup}

We define $C(h)$
\begin{align*}
    C(h)&: \R_{\geq0} \times X \rightarrow X \nonumber
\end{align*}
$\{C(h)\}_{h\geq 0}$ forms a one parameter semigroup if the the following requirements are met:

\qquad Associativity:
\begin{equation}
    \forall h,s\geq0: C(h+s) = C(h)C(s) \label{eq:associativity}
\end{equation}
\qquad Identity: 
\begin{equation}
    C(0) = I \label{eq:identity}
\end{equation}

In the context of global and local linear maps, the parameter \( t \) referred to specific time points in the system’s evolution, and \( h \) represented the time step or interval over which the system evolved. 

In this context, $h$ describes the continuous evolution of the operator family $\{C(h)\}_{h\geq 0}$, functioning similarly to the time step 
$h$ in the discrete-time models. However, here 
$h$ represents the time duration over which the system evolves, extending the discrete case into a continuous framework.

\subsubsection{$C_0$-semigroup}

Continuity for one-parameter semigroups is defined by the following properties:

\paragraph{Uniform Continuity:}
A one-parameter semigroup \( \{C(h)\}_{h \geq 0} \) is uniformly continuous if:
\begin{align}
    \lim_{h \to 0^+} || C(h) x - x || = 0, \quad \forall x \in X \label{eq:uniformity}
\end{align}
Uniform continuity implies that the operator family \( C(h) \) behaves "smoothly" as \( h \to 0^+ \), ensuring that the system’s evolution is well-behaved and bounded. 

The linear operator \( A \) is the infinitesimal generator of a uniformly continuous semigroup if and only if \( A \) is a \textit{bounded linear operator}. In this case, the infinitesimal generator \( A \) of the semigroup \( C(h) \) must satisfy:
\begin{align}
    C(h) = e^{Ah} := \sum_{k=0}^{\infty} \frac{A^k}{k!} h^k \label{eq:generator}
\end{align}
This exponential form provides a strong connection to the classical matrix exponential, commonly seen in linear systems.

\paragraph{Strong Continuity:}
If the semigroup is not uniformly continuous but still exhibits strong continuity, it is called a \textit{$C_0$-semigroup}. This requires that for any \( x \in X \):
\begin{align}
    \lim_{h \to 0^+} || C(h) x - x || = 0, \quad \forall x \in X \label{eq:strong_continuity}
\end{align}
This property ensures that the semigroup \( C(h) \) is \textit{strongly continuous}, even if the system's evolution is not uniformly bounded.

For a strongly continuous semigroup, we define the \textit{infinitesimal generator} \( A \), acting on the domain \( D(A) \), as:
\begin{equation}
    Ax =  \lim_{h\rightarrow0^+} \frac{1}{h} (C(h) - I)x \label{eq:continuity}
\end{equation}
The limit must exist for \( x \in D(A) \), but unlike the uniformly continuous case, the generator \( A \) need not be bounded.

In this more general case, the semigroup \( C(h) \) is still governed by the operator \( A \), but the series \( e^{Ah} \) need not converge if \( A \) is unbounded.

The $C_0$-semigroup provides a strong foundation for understanding time evolution in terms of linear operators, but it does not fully capture the role of locality or the higher-order terms in systems like \( X_t \). We now extend the $C_0$-semigroup to the more general $C_n$-semigroup, which accounts for higher-order effects via finite differences and their infinitesimal generators.

\subsection{$C_n$-semigroup}

The $C_0$-semigroup can be viewed as a special case of a more general continuous one-parameter semigroup, the nonlinear $C_n$-semigroup. This semigroup's infinitesimal generator expresses higher-order effects through the limit of ever smaller finite differences.

We guide and justify the definition of such a semigroup based on the fact that the exponential of the infinitesimal generator yields a well-known generalization of the Taylor series—using the infinitesimal limit of finite differences—known as the Hille series.

The key advantage of the $C_n$-semigroup and its infinitesimal generator is that they carry intuition regarding continuity and nonlinearity into a more pragmatic operator-theoretic framework. This makes the $C_n$-semigroup particularly well-suited for investigating local dynamics and the role of locality in nonlinear systems.

\subsubsection{The Hille Operator for Nonlinear Semigroups}

We can relate the forward differences to the semigroup where:
$$
    \Delta_h^1 x(t) = x(t+h) - x(t) = C_1(h) x(t) - x(t).
$$ 

For higher-order differences, the difference operator series is:
\begin{align}
    \Delta_h^n x(t) &= \sum_{k=0}^n (-1)^{n-k} {n \choose k} x(t+kh) \nonumber \\
    &= \sum_{k=0}^n (-1)^{n-k} {n \choose k} C_k(h) x(t) \label{eq:delta_c}
\end{align}

If we define the $n$-order generator of a strongly continuous nonlinear semigroup:
$$
    A^n x = \lim_{h\rightarrow0^+} \frac{1}{h^n} (C_n(h) - I) x,
$$
then the higher-order generators, $\{A^n\}_{n\geq2}$, describe higher-order terms of the semigroup, capturing corrections to the first-order dynamics \( A^1 \).

We can express local continuity of the nonlinear $C_n$-semigroup as the matrix exponential of the $n$-order generators,
\begin{align*}
     C_n(h) = e^{Ah} &:= \sum_{k=0}^n \frac{h^k}{k!} A^k\\
     &:= \lim_{h\rightarrow0^+} \sum_{k=0}^n \frac{h^k}{k!} \frac{C^n(h) - I}{h^n} 
\end{align*}
In the limit
\begin{align*}
    \lim_{h\rightarrow0^+} C_n(h) &= \lim_{h\rightarrow0^+} \sum_{k=0}^n \frac{h^k}{k!} A^k \\
    H &= \lim_{h\rightarrow0^+} \sum_{k=0}^n \frac{h^k}{k!} \frac{C^n(h) - I}{h^n}
\end{align*}
We call $H$ the \textit{Hille operator}. Upon application, \( Hx(t) \), this operator yields a special case of the Hille series:
\begin{align*}
    H x(t) &= \lim_{h\rightarrow0^+} \sum_{k=0}^n \frac{h^k}{k!} \frac{C^k(h) - I}{h^k} x(t)\\
     &= \lim_{h\rightarrow0^+} \sum_{k=0}^n \frac{h^k}{k!} \frac{C^k(h)x(t) - x(t)}{h^k}\\
      &= \lim_{h\rightarrow0^+} \sum_{k=0}^n \frac{h^k}{k!} \frac{x(t+kh) - x(t)}{h^k}\\
    x(t+h) &= \lim_{h\rightarrow0^+} \sum_{k=0}^\infty \frac{\Delta_h^k x(t) }{k!} 
\end{align*}
\begin{tcolorbox}[colback=gray!10, colframe=gray!40, width=\textwidth, boxrule=0pt, sharp corners, leftrule=4mm]

    The notation can be a little confusing here. In our formulation we identify $h$ as both the span over which we apply the operator and the higher-order infinitesimals. This offers the simplification of the time dependent and infinitesimal terms to yield the local expansion at $t$. Whereas in the more general Hille series,
    $$
    x(a+t) = \lim_{h\rightarrow0^+}\sum_{k=0}^\infty \frac{t^k}{k!} \frac{\Delta^k_h x(a)}{h^k} 
    $$
    $t$ is the independent time parameter and $h$ is the infinitesimal step size. The simplification merely arises from the fact that, for a one parameter semigroup, $a=t\in \R_{\geq0}$

\end{tcolorbox}

 $H$ provides a way to generalize the semigroup dynamics to include higher-order effects via the Hille like expansion, defining the locally continuous action of a one-parameter nonlinear semigroup. This generalization, where each term captures the evolution of the semigroup through increasingly higher-order dynamics, allows us to track the local nonlinear behavior of the system, much like a Taylor expansion tracks higher-order terms in a polynomial.

If the $C_n$-semigroup possesses uniform continuity for $t\in\R_{\geq0}$, then,
\begin{align*}
    C_n(t) = e^{(e^{Ah})t} &:= \sum_{m=0}^\infty \frac{t^m}{m!} \left(\lim_{h\rightarrow0^+} \sum_{k=0}^n \frac{h^k}{k!} A^k\right)^m\\
\end{align*}
where $t$ is the global parameter governing the semigroup's evolution in time and the inner sum $\sum_{k=0}^n \frac{h^k}{k!} A^k$ represents the local action up to $n$-orders over an infinitesimal $h$. 

The recursive application of the matrix exponential provides a mechanism for capturing higher-order corrections to local continuity. Uniform continuity is obtained if the generators $A^n$ remain bounded, extending the notion of smooth evolution across higher-order dynamics.

\begin{tcolorbox}[colback=gray!10, colframe=gray!40, width=\textwidth, boxrule=0pt, sharp corners, leftrule=4mm]
    \textbf{Aside:}
    In a way the recursive operation of the exponential extends the semigroup across different measures. The first application, $e^{Ah}$, yields the local semigroup action at a set of measure-$0$. The second, $e^{(e^{Ah}) t}$ yields the action across  a trajectory set of measure-$1$, We could apply once more for a manifold of measure-$2$, 
    $$
        \exp(A, h, 3) =e^{(e^{(e^{Ah})t_1})t_2}
    $$
    This could be explored elsewhere though as a formalization of multiparameter semigroups and investigate connections to concepts in other fields like the exponential map in Riemannian geometry, which takes a tangent vector (local direction) and maps it to a geodesic (global trajectory). This could provide a nice supplemental perspective when investigating locality and curvature across dimensions.
\end{tcolorbox}

Taking the limit as $n \rightarrow \infty$ allows us to transition from discrete higher-order semigroups to continuous generators, capturing nonlinear interactions across all scales.

\subsubsection{$C_\infty$-semigroup}
If taking $n$ at the limit $\lim\rightarrow\infty$ we argue for a continuous integration over the set of local operators,
\begin{align*}
   \lim_{n\rightarrow\infty} \lim_{h\rightarrow0^+} \sum_{k=0}^n \frac{h^k}{k!} A^k \ &=\lim_{h\rightarrow0^+} \lim_{n\rightarrow\infty}  \sum_{k=0}^n \frac{h^k}{k!} A^k \\
    \lim_{n\rightarrow\infty} \lim_{h\rightarrow0^+} \sum_{k=0}^n \frac{h^k}{k!} A^k \ &=\lim_{h\rightarrow0^+} \int_0^\infty \frac{h^r}{\Gamma(r+1)} A^r dr
\end{align*}

Consider the continuous forward difference operator,
\begin{align*}
    \Delta^r_h x(t) = \sum_{k=0}^\infty (-1)^k \frac{\Gamma(r+1)}{\Gamma(k+1)\Gamma(r-k+1)} x(t+kh)
\end{align*}
We redefine $A^n$ in terms of the continuous order parameter $r\in\R_{\geq0}$,
\begin{align*}
    A^r x(t) = \lim_{h\rightarrow0^+} \frac{1}{h^r} (C^r(h) - I) x(t)
\end{align*}
If $C^r(h)x(t) = x(t+rh)$, then
\begin{align*}
    A^r x(t) &= \lim_{h\rightarrow0^+} \frac{1}{h^r} (C^r(h) - I) x(t)\\
    &= \lim_{h\rightarrow0^+} \frac{1}{h^r} (C^r(h)x(t) - x(t))  \\
    &= \lim_{h\rightarrow0^+} \frac{1}{h^r} (x(t+rh) - x(t)) \\
    &= \lim_{h\rightarrow0^+} \frac{\Delta^r_h }{h^r}x(t)
\end{align*}
The matrix exponential
\begin{align*}
    A(h) &= e^{\int Ah}\\
    &:= \int_0^\infty \frac{h^r}{\Gamma(r+1)} A^r dr\\
    &:= \int_0^\infty \frac{h^r}{\Gamma(r+1)} \frac{\Delta^r_h }{h^r} dr\\
    &:= \int_0^\infty \frac{\Delta^r_h}{\Gamma(r+1)} dr
\end{align*}
yields the strongly nonlinear Hille operator in the limit,
\begin{align*}
    H &= \lim_{h\rightarrow0^+}  e^{\int Ah}\\
    &=\lim_{h\rightarrow0^+} \int_0^\infty \frac{\Delta^r_h}{\Gamma(r+1)} dr
\end{align*}
which we take as the infinitesimal generator for the $C_\infty$-semigroup
\begin{align*}
    C_\infty(t) &= e^{Ht}\\
    &:= \sum_{n=0}^\infty \frac{t^n}{n!}  \int_0^\infty \frac{\lim_{h\rightarrow0^+} \Delta^r_h}{\Gamma(r+1)} dr
\end{align*}

 
\subsection{The resolvents of the $C_n$ and $C_\infty$ semigroups}

The resolvent 
$$
    R(\lambda, H) = \lim_{h\rightarrow0^+}(\lambda I - H)^{-1}
$$
defines the spectrum and growth behavior of the semigroup.

We are interested in the resolvent set that provides a well defined inverse,
$$
    \rho(H) = \left\{\lambda\in\C: \lim_{h\rightarrow0^+} \left( \lambda I - H \right) ^{-1}\right\}
$$ 
For the discrete case we have,
\begin{align*}
     R(\lambda, H_n) &= \lim_{h\rightarrow0^+}\left( \lambda I - H\right) ^{-1}\\
     &= \left( \lambda I - \lim_{h\rightarrow0^+} \sum_n \frac{\Delta^n_h}{n!}\right) ^{-1} &n\in\N
\end{align*}
and for the continuum,
\begin{align*}
    R(\lambda, H_\infty ) &= \left( \lambda I - \lim_{h\rightarrow0^+} \int_0^\infty \frac{\Delta^r_h}{\Gamma(r+1)} dr\right) ^{-1} &r\in\R_{\geq0}
\end{align*}
We need $\lambda \in \rho(H)$ such that
\begin{align*}
    \rho(H) &=\C /  \sigma\left(H\right)\\
\end{align*}
where $\sigma(h)$ represents the spectrum of the Hille operator generating the semigroup 

The resolvent set $\rho(H)$ is crucial for understanding the stability and long-term behavior of the semigroup. In particular, the spectral bound, 
\begin{align*}
    s(H) = \sup \{\Re(\lambda): \lambda \in \sigma(H) \}
\end{align*}
governs the growth rate of the semigroup. If 
$$
    s(H) \leq 0 
$$
then the semigroup is stable and contracts over time, while positive values indicate exponential growth.

\section{$C_n$-semigroup Operating On Matrices}

A relatively trivial right translation of $C_n$-semigroup allows meaningful operation on state space matrices of trajectories in $X$. We reframe 
\begin{align*}
    x(t+kh) = C_k(h) x(t) 
\end{align*}
as a right translation
\begin{align*}
    X(t+kh) = X(t) C_k(h)
\end{align*}
of the semigroup.

We have,
\begin{align*}
    \Delta_h^n X(t) = \sum_{k=0}^n (-1)^{n-k} {n \choose k} X(t+kh)
\end{align*}
and define the right $n$-order generator of a strongly continuous nonlinear semigroup:
\begin{align*}
    X A^n = \lim_{h\rightarrow0^+} \frac{1}{h^n} X (C_n(h) - I)
\end{align*}
The matrix exponential requiring right multiplication is
\begin{align*}
    C(h) = e^{hA} &:= \sum_{k=0}^n  A^k \frac{h^k}{k!}\\
     &:= \lim_{h\rightarrow0^+} \sum_{k=0}^n  \frac{C^n(h) - I}{h^n} \frac{h^k}{k!}
\end{align*}
where the corresponding Hille operator
\begin{align*}
    \lim_{h\rightarrow0^+} C(h) &= \lim_{h\rightarrow0^+} \sum_{k=0}^n A^k \frac{h^k}{k!}  \\
    H &= \lim_{h\rightarrow0^+} \sum_{k=0}^n \frac{C^n(h) - I}{h^n} \frac{h^k}{k!} \\
\end{align*}
which upon right application gives, 
\begin{align*}
    X(t) H &= \lim_{h\rightarrow0^+} \sum_{k=0}^n X(t)  \frac{C^k(h) - I}{h^k} \frac{h^k}{k!} \\
     &= \lim_{h\rightarrow0^+} \sum_{k=0}^n  \frac{X(t) C^k(h) - X(t)}{h^k} \frac{h^k}{k!}\\
      &= \lim_{h\rightarrow0^+} \sum_{k=0}^n  \frac{X(t+kh) - X(t)}{h^k} \frac{h^k}{k!}\\
    X(t+h) &= \lim_{h\rightarrow0^+} \sum_{k=0}^\infty \frac{\Delta_h^k X(t) }{k!} 
\end{align*}
Uniform continuity is given by
\begin{align*}
    C_n(t) = e^{t(e^{hA})} &:= \sum_{m=0}^\infty  \left(\sum_{k=0}^n  A^k \frac{h^k}{k!}\right)^m \frac{t^m}{m!}\\
\end{align*}
As before we consider the achievement of the limit of $n$ as a continuous extension,
\begin{align*}
    H &= \lim_{h\rightarrow0^+}  e^{\int hA}\\
    &= \int_0^\infty  A^r \frac{h^r}{\Gamma(r+1)} dr\\
    &=\lim_{h\rightarrow0^+} \int_0^\infty \frac{\Delta^r_h}{\Gamma(r+1)} dr
\end{align*}

\end{document}